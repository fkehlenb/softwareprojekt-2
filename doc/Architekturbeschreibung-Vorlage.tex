\documentclass[enabledeprecatedfontcommands,fontsize=12pt,paper=a4,twoside]{scrartcl}


\newcommand{\grad}{\ensuremath{^{\circ}} }
\renewcommand{\strut}{\vrule width 0pt height5mm depth2mm}

\usepackage[utf8]{inputenc}
\usepackage[T1]{fontenc}
\usepackage[final]{pdfpages}
% obere Seitenränder gestalten können
\usepackage{fancyhdr}
\usepackage{moreverb}
% Graphiken als jpg, png etc. einbinden können
\usepackage{graphicx}
\usepackage{stmaryrd}
% Floats Objekte mit [H] festsetzen
\usepackage{float}
% setzt URL's schön mit \url{http://bla.laber.com/~mypage}
\usepackage{url}
% Externe PDF's einbinden können
\usepackage{pdflscape}
% Verweise innerhalb des Dokuments schick mit " ... auf Seite ... "
% automatisch versehen. Dazu \vref{labelname} benutzen
\usepackage[ngerman]{varioref}
\usepackage[ngerman]{babel}
\usepackage{ngerman}
% Bibliographie
\usepackage{bibgerm}
% Tabellen
\usepackage{tabularx}
\usepackage{supertabular}
\usepackage[colorlinks=true, pdfstartview=FitV, linkcolor=blue,
            citecolor=blue, urlcolor=blue, hyperfigures=true,
            pdftex=true]{hyperref}
\usepackage{bookmark}

\hyphenation{Arbeits-paket}

% Damit Latex nicht zu lange Zeilen produziert:
\sloppy
%Uneinheitlicher unterer Seitenrand:
%\raggedbottom

% Kein Erstzeileneinzug beim Absatzanfang
% Sieht aber nur gut aus, wenn man zwischen Absätzen viel Platz einbaut
\setlength{\parindent}{0ex}

% Abstand zwischen zwei Absätzen
\setlength{\parskip}{1ex}

% Seitenränder für Korrekturen verändern
\addtolength{\evensidemargin}{-1cm}
\addtolength{\oddsidemargin}{1cm}

\bibliographystyle{gerapali}

% Lustige Header auf den Seiten
  \pagestyle{fancy}
  \setlength{\headheight}{70.55003pt}
  \fancyhead{}
  \fancyhead[LO,RE]{Software--Projekt 2\\ WiSe 2019/2020
  \\Architekturbeschreibung}
  \fancyhead[LE,RO]{Seite \thepage\\\slshape \leftmark\\\slshape \rightmark}

%
% Und jetzt geht das Dokument los....
%

\begin{document}

% Lustige Header nur auf dieser Seite
  \thispagestyle{fancy}
  \fancyhead[LO,RE]{ }
  \fancyhead[LE,RO]{Universität Bremen\\FB 3 -- Informatik\\
  Prof. Dr. Rainer Koschke \\TutorIn: Marcel Steinbeck}
  \fancyfoot[C]{}

% Start Titelseite
  \vspace{3cm}

  \begin{minipage}[H]{\textwidth}
  \begin{center}
  \bf
  \Large
  Software--Projekt 2 WiSe 2019/2020\\
  \smallskip
  \small
  VAK 03-BA-901.02\\
  \vspace{3cm}
  \end{center}
  \end{minipage}
  \begin{minipage}[H]{\textwidth}
  \begin{center}
  \vspace{1cm}
  \bf
  \Large Architekturbeschreibung\\
  \vfill
  \end{center}
  \end{minipage}
  \vfill
  \begin{minipage}[H]{\textwidth}
  \begin{center}
  \sf
  \begin{tabular}{lr}
  Liam Hurwitz & hurwitz@tzi.de \\
  xxxx xxxxxxxx & xxxx@tzi.de \\
  \end{tabular}
  \\ ~
  \vspace{2cm}
  \\
  \it Abgabe: TT. Monat JJJJ --- Version 1.0\\ ~
  \end{center}
  \end{minipage}

% Ende Titelseite

% Start Leerseite

\newpage

  \thispagestyle{fancy}
  \fancyhead{}
  \fancyhead[LO,RE]{Software--Projekt \\  2019/2020
  \\Architekturbeschreibung}
  \fancyhead[LE,RO]{Seite \thepage\\\slshape \leftmark\\~}
  \fancyfoot{}
  \renewcommand{\headrulewidth}{0.4pt}
  \tableofcontents

\newpage

  \fancyhead[LE,RO]{Seite \thepage\\\slshape \leftmark\\\slshape \rightmark}


%%%%%%%%%%%%%%%%%%%%%%%%%%%%%%%%%%%%%%%%%%%%%%%%%%%%%%%%%%%%%%%%%%%%%%%%
\section*{Version und Änderungsgeschichte}

{\em Die aktuelle Versionsnummer des Dokumentes sollte eindeutig und gut zu
identifizieren sein, hier und optimalerweise auf dem Titelblatt.}

\begin{tabular}{ccl}
Version & Datum & Änderungen \\
\hline
0.1 & TT.MM.JJJJ & Dokumentvorlage als initiale Fassung kopiert \\
0.2 & TT.MM.JJJJ & \ldots \\
\ldots
\end{tabular}


%%%%%%%%%%%%%%%%%%%%%%%%%%%%%%%%%%%%%%%%%%%%%%%%%%%%%%%%%%%%%%%%%%%%%%%%
\section{Einführung}

\subsection{Zweck}

  {\em Was ist der Zweck dieser Architekturbeschreibung? Wer sind
  die LeserInnen?}

\subsection{Status}



\subsection{Definitionen, Akronyme und Abkürzungen}


\subsection{Referenzen}

\subsection{Übersicht über das Dokument}


\section{Globale Analyse}
\label{sec:globale_analyse}

{\it Hier werden Einflussfaktoren aufgezählt und bewertet sowie Strategien
zum Umgang mit interferierenden Einflussfaktoren entwickelt.}

\subsection{Einflussfaktoren}
\label{sec:einflussfaktoren}
{\it Hier sind Einflussfaktoren gefragt, die sich auf die Architektur
  beziehen. Es sind ausschließlich architekturrelevante
  Einflussfaktoren, und nicht z.\,B.\ solche, die lediglich einen
  Einfluss auf das Projektmanagement haben. Fragt Euch also bei jedem
  Faktor: Beeinflusst er wirklich die Architektur? Macht einen
  einfachen Test: Wie würde die Architektur aussehen, wenn ihr den
  Einflussfaktor E berücksichtigt? Wie würde sie aussehen, wenn Ihr E nicht
  berücksichtigt? Kommt in beiden Fällen dieselbe Architektur heraus,
  dann kann der Einflussfaktor nicht architekturrelevant sein.

  Es geht hier um Einflussfaktoren, die
  \begin{enumerate}
  \item sich über die Zeit ändern,
  \item viele Komponenten betreffen,
  \item schwer zu erfüllen sind oder
  \item mit denen man wenig Erfahrung hat.
  \end{enumerate}
  Die Flexibilität und Veränderlichkeit müssen ebenfalls charakterisiert werden.
  \begin{enumerate}
  \item Flexibilität: Könnt Ihr den Faktor zum jetzigen Zeitpunkt beeinflussen?
  \item Veränderlichkeit: ändert der Faktor sich später durch äußere Einflüsse?
\end{enumerate}

  Unter Auswirkungen sollte dann beschrieben werden, {\em wie} der
  Faktor {\em was} beeinflusst. Das können sein:
  \begin{itemize}
  \item andere Faktoren
  \item Komponenten
  \item Operationsmodi
  \item Designentscheidungen (Strategien)
  \end{itemize}

  Verwendet eine eindeutige Nummerierung der Faktoren, um sie auf den
  Problemkarten einfach referenzieren zu können.  }

Im folgendem Teil erläutern wir die verschienden Akteure
Prozesskettenadministrator, Logitisker, Techniker, Transport und
Administrator. Nach dem Auflisten der Aufgaben der Akteure \textbf{1.
Prozesskettenadministrator} * \textbf{Aufgabe im SFB:} 1. Prozessketten
anlegen 2. Prozessketten instanziieren 3. Den Überblick behalten`` *
\textbf{Sein Problem:} 1. Kein direktes Feedback von den
Experimentalstationen 2. Keine Abbildung von Prozessketten in der DB *
Was wünscht er sich? 1. Übersicht über aktuell laufende
Prozessketteninstanzen (und deren Status) * \textbf{Was muss er tun?} *
Übersicht über alle Prozessschritte 1. Anlegen neuer Prozessschritte *
Auswahl von vordefinierten Abläufen (Zustandsautomat) * Auswahl von dyn.
Abläufen (O«Dynamische Zustandsautomaten») * Eingabe von
voraussichtlicher Prozessschrittdauer * Definition von
Prozessschrittparametern (Name, Einheit) * Zuordnung von
Experimentierstation * Setzen von Attributen wie ein färbende,
Präparationsart 3. Löschen bestehender Prozessschritte 2. Prüfen (Auf
Korrektheit) 3. Bearbeiten von nicht gestarteten Prozessschritten 4.
Hervorhebung der Prozessschritte, die im Zustand auf kaputt gesetzt sind

\begin{verbatim}
* Übersicht über dynamische Abläufe (O«Dynamische Zustandsautomaten»)
    1. Anlegen von Zustandsautomaten
    2. Löschen von Zustandsautomaten
    3. Bearbeiten von unbenutzten Zustandsautomaten

* Übersicht über alle Prozessketten
    1. Anlegen von neuen Prozessketten aus bestehenden Prozessschritten
    1. Löschen von bestehenden Prozessketten
    1. Bearbeiten von bestehenden Prozessketten (nicht durchgeführt oder freigegeben)

* Übersicht über Aufträge
    * Anlegen von Aufträgen
        * Manuelles Setzen der Prozess(schritt)parameter
        * Import der Prozess(schritt)parameter (O«Parameter.Import»)
    * Freigabe von Aufträgen
        * Logistiker muss daraufhin Proben zuordnen
    * Stoppen von Aufträgen
        * Aktueller Prozessschritt wird beendet
        * Aufträge verschwinden aus der Übersicht der Technologen
        * Transport bringt Proben ins Lager (O«Transporteur»)
    * Löschen von Aufträgen
    * Bearbeiten von Aufträgen (wenn noch nicht freigegeben)
    * Priorisierung der Aufträgen

* Übersicht über Arbeitsauslastung der Experimentierstationen
  > siehe (O«UI.Auslastung»)
  > 
* Export des Protokolls nach JSON, so dass eine maschinelle Auswertung möglich ist 
    
* Prozessketteninstanzen
    1. Priorisieren
    2. Anlegen und Probeneigenschaften festlegen (Legierung + Wärmebehandlung)
\end{verbatim}

Es gibt Prozessketten Templates

\begin{quote}
Ich will effizient den Überblick über die laufende
Prozessketteninstanzen behalten und direkt Feedback über von den
Experimentalstationen erhalten. Prozesskette hätte ich gerne in der
Software abgebi
\end{quote}

\subsection{Probleme und Strategien}
\label{sec:strategien}

{\it Aus einer Menge von Faktoren ergeben sich Probleme, die nun in
  Form von Problemkarten beschrieben werden. Diese resultieren
  z.\,B. aus
  \begin{itemize}
  \item Grenzen oder Einschränkungen durch Faktoren
  \item der Notwendigkeit, die Auswirkung eines Faktors zu begrenzen
  \item der Schwierigkeit, einen Produktfaktor zu erfüllen, oder
  \item der Notwendigkeit einer allgemeinen Lösung zu globalen
    Anforderungen.
  \end{itemize}
  Dazu entwickelt Ihr Strategien, um mit den identifizierten Problemen
  umzugehen.

  Achtet auch hier darauf, dass die Probleme und Strategien wirklich
  die Architektur betreffen und nicht etwa das Projektmanagement. Die
  Strategien stellen im Prinzip die Designentscheidungen dar. Sie
  sollten also die Erklärung für den konkreten Aufbau der
  verschiedenen Sichten liefern.}


\textit{Beschreibt möglichst mehrere Alternativen und gebt
  an, für welche Ihr Euch letztlich aus welchem Grunde entschieden
  habt. Natürlich müssen die genannten Strategien in den folgenden
  Sichten auch tatsächlich umgesetzt werden!}

\textit{Ein sehr häufiger Fehler ist es, dass SWP-Gruppen
  arbeitsteilig vorgehen: die eine Gruppe schreibt das Kapitel zur
  Analyse von Faktoren und zu den Strategien, die andere Gruppe
  beschreibt die diversen Sichten, ohne dass diese beiden Gruppen sich
  abstimmen. Natürlich besteht aber ein Zusammenhang zwischen den
  Faktoren, Strategien und Sichten. Dieser muss erkennbar sein, indem
  sich die verschiedenen Kapitel eindeutig aufeinander beziehen.}

\section{Konzeptionelle Sicht}
\label{sec:konzeptionell}

{\it Diese Sicht beschreibt das System auf einer hohen Abstraktionsebene,
d.\,h. mit sehr starkem Bezug zur Anwendungsdomäne und den geforderten
Produktfunktionen und \linebreak-attributen. Sie legt die Grobstruktur fest,
ohne gleich in die Details von spezifischen Technologien abzugleiten.
Sie wird in den nachfolgenden Sichten konkretisiert und verfeinert. Die
konzeptionelle Sicht wird mit {UML}-Komponentendiagrammen visualisiert.}

\section{Modulsicht}
\label{sec:modulsicht}

{\it
Diese Sicht beschreibt den statischen Aufbau des Systems mit Hilfe von
Modulen, Subsystemen, Schichten und Schnittstellen.
Diese Sicht ist hierarchisch, d.\,h. Module werden in Teilmodule
zerlegt. Die Zerlegung endet bei Modulen, die ein klar umrissenes
Arbeitspaket für eine Person darstellen und in einer Kalenderwoche
implementiert werden können. Die Modulbeschreibung der Blätter dieser
Hierarchie muss genau genug und ausreichend sein, um das Modul
implementieren zu können.

Die Modulsicht wird durch {UML}-Paket- und Klassendiagramme visualisiert.

Die Module werden durch ihre Schnittstellen beschrieben.
Die Schnittstelle eines Moduls $M$ ist die Menge aller Annahmen, die
andere Module über $M$ machen dürfen, bzw.\ jene Annahmen, die $M$
über seine verwendeten Module macht (bzw. seine Umgebung, wozu auch
Speicher, Laufzeit etc.\ gehören).
Konkrete Implementierungen dieser Schnittstellen sind das Geheimnis des Moduls
und können vom Programmierer festgelegt werden. Sie sollen hier
dementsprechend nicht beschrieben werden.

Die Diagramme der Modulsicht sollten die zur Schnittstelle gehörenden Methoden
enthalten. Die Beschreibung der einzelnen Methoden (im Sinne der Schnittstellenbeschreibung)
geschieht allerdings per Javadoc im zugehörigen Quelltext. Das bedeutet, dass Ihr
für alle Eure Module Klassen, Interfaces und Pakete erstellt und sie mit den Methoden der
Schnittstellen verseht. Natürlich noch ohne Methodenrümpfe bzw.\ mit minimalen Rümpfen.
Dieses Vorgehen vereinfacht den Schnittstellenentwurf und stellt Konsistenz sicher.

Jeder Schnittstelle liegt ein
Protokoll zugrunde. Das Protokoll beschreibt die Vor- und
Nachbedingungen der Schnittstellenelemente. Dazu gehören die erlaubten
Reihenfolgen, in denen Methoden der Schnittstelle aufgerufen werden
dürfen, sowie Annahmen über Eingabeparameter und Zusicherungen über
Ausgabeparameter. Das Protokoll von Modulen wird in der Modulsicht beschrieben.
Dort, wo es sinnvoll ist, sollte es mit Hilfe von Zustands- oder
Sequenzdiagrammen spezifiziert werden. Diese sind dann einzusetzen, wenn der
Text allein kein ausreichendes Verständnis vermittelt (insbesondere
bei komplexen oder nicht offensichtlichen Zusammenhängen).

Der Bezug zur konzeptionellen Sicht muss klar ersichtlich sein. Im
Zweifel sollte er explizit erklärt werden. Auch für diese Sicht muss
die Entstehung anhand der Strategien erläutert werden.
}

\section{Datensicht}
\label{sec:datensicht}

{\it Hier wird das der Anwendung zugrundeliegende Datenmodell
  beschrieben. Hierzu werden neben einem erläuternden Text auch ein
  oder mehrere {UML}-Klassendiagramme verwendet. Das hier beschriebene
  Datenmodell wird u.\,a. jenes der Anforderungsspezifikation enthalten,
  allerdings mit implementierungsspezifischen Änderungen und
  Erweiterungen. Siehe die gesonderten Hinweise.}

\section{Ausführungssicht}

\label{sec:ausfuehrung}

{\it
Die Ausführungssicht beschreibt das Laufzeitverhalten. Hier
werden die Laufzeitelemente aufgeführt und beschrieben, welche Module
sie zur Ausführung bringen. Ein Modul kann von mehreren
Laufzeitelementen zur Laufzeit verwendet werden. Die Ausführungssicht
beschreibt darüber hinaus, welche Laufzeitelemente spezifisch
miteinander kommunizieren. Zudem wird bei verteilten Systemen
(z.\,B. Client-Server-Systeme) dargestellt, welche Module von welchen
Prozessen auf welchen Rechnern ausgeführt werden.}


\section[Zusammenhänge zwischen Anwendungsfällen und Architektur]{Zusammenhänge zwischen Anwendungsfällen und Architektur\sectionmark{Zusammenhänge AF u. Architektur}}
\sectionmark{Zusammenhänge AF u. Architektur}
\label{sec:anwendungsfaelle}

{\it In diesem Abschnitt sollen Sequenzdiagramme mit Beschreibung(!)
  für zwei bis drei von Euch ausgewählte
    Anwendungsfälle
  erstellt werden. Ein Sequenzdiagramm beschreibt den
  Nachrichtenverkehr zwischen allen Modulen, die an der Realisierung
  des Anwendungsfalles beteiligt sind.  Wählt die
    Anwendungsfälle so, dass nach Möglichkeit alle Module Eures
    entworfenen Systems in mindestens einem Sequenzdiagramm
    vorkommen. Falls Euch das nicht gelingt, versucht möglichst viele
    und die wichtigsten Module abzudecken. }

\section{Evolution}

\label{sec:evolution}

{\it
  Beschreibt in diesem Abschnitt, welche Änderungen Ihr
  vornehmen müsst, wenn sich Anforderungen oder Rahmenbedingungen
  ändern. Insbesondere würden hierbei die in der
  Anforderungsspezifikation unter "`Ausblick"' genannten
  Punkte behandelt werden.}

\dots


\end{document}


%%% Local Variables:
%%% mode: latex
%%% mode: reftex
%%% mode: flyspell
%%% ispell-local-dictionary: "de_DE"
%%% TeX-master: t
%%% End:
